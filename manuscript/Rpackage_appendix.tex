\clearpage
\makeatletter
\efloat@restorefloats
\makeatother


\begin{appendix}
\hypertarget{transforming-an-ordered-probability-vector-to-the-real-line}{%
\section{Transforming An Ordered Probability Vector To The Real
Line}\label{transforming-an-ordered-probability-vector-to-the-real-line}}

Since we choose the multivariate normal as proposal distribution, the
mapping between the proposal and target distribution requires us to move
\(\boldsymbol{\theta}\) to the real line. Crucially, the transformation
needs to retain the ordering of the parameters, that is, it needs to
take into account the lower bound \(l_k\) and the upper bound \(u_k\) of
each \(\theta_k\). To achieve this goal, \textbf{multibridge} uses a
probit transformation as proposed in Sarafoglou et al. (2020) which
subsequently transforms the elements in \(\boldsymbol{\theta}\) moving
from its lowest to its highest value. In the binomial model, we move all
elements in \(\boldsymbol{\theta}\) to the real line and thus construct
a new vector \(\boldsymbol{y} \in \mathbb{R}^{K}\). For multinomial
models it follows from the sum-to-one constraint that the vector
\(\boldsymbol{\theta}\) is completely determined by its first \(K - 1\)
elements, where \(\theta_K\) is defined as
\(1 - \sum_{k = 1}^K \theta_k\). Hence, for multinomial models we will
only consider the first \(K - 1\) elements of \(\boldsymbol{\theta}\)
and we will transform them to \(K - 1\) elements of a new vector
\(\boldsymbol{y} \in \mathbb{R}^{K - 1}\).

Let \(\phi\) denote the density of a normal variable with a mean of zero
and a variance of one, \(\Phi\) denote its cumulative density function,
and \(\Phi^{-1}\) denote the inverse cumulative density function. Then
for each element \(\theta_k\), the transformation is
\[\xi_k = \Phi^{-1}\left(\frac{\theta_k - l_k}{u_k - l_k}\right),\] The
inverse transformation is given by
\[\theta_k = (u_k - l_k) \Phi(\xi_k) + l_k.\]

To perform the transformations, we thus need to determine the lower
bound \(l_k\) and the upper bound \(u_k\) of each \(\theta_k\). Assuming
\(\theta_{k-1} < \theta_{k}\) for \(k \in \{1 \cdots, K\}\) the lower
bound for any element in \(\boldsymbol{\theta}\) is defined as

\begin{align*}
  l_k = \left.
  \begin{cases}
      0 & \text{if } k = 1 \\
      \theta_{k - 1} & \text{if } 1 < k < K.
  \end{cases}
    \right.
\end{align*}

This definition holds for both binomial models and multinomial models.
Differences in these two models appear only when determining the upper
bound for each parameter. For binomial models, the upper bound for each
\(\theta_k\) is simply \(1\). For multinomial models, however, due to
the sum-to-one constraint the upper bounds depend on the values of
smaller elements as well as on the number of remaining larger elements
in \(\boldsymbol{\theta}\). To be able to determine the upper bounds, we
represent \(\boldsymbol{\theta}\) as unit-length stick which we
subsequently divide into \(K\) elements (Frigyik, Kapila, \& Gupta,
2010; Stan Development Team, 2020). By using this so-called
stick-breaking method we can define the upper bound for any \(\theta_k\)
as follows:

\begin{align}
\label{Eq:upperBound}
  u_k = \left.
  \begin{cases}
      \cfrac{1}{K} & \text{if } k = 1 \\
      \cfrac{1 - \sum_{i < k} \theta_i}{ERS} & \text{if } 1 < k < K,
  \end{cases}
    \right.
\end{align} where \(1 - \sum_{i < k} \theta_i\) represents the length of
the remaining stick, that is, the proportion of the unit-length stick
that has not yet been accounted for in the transformation. The elements
in the remaining stick are denoted as \(ERS\), and are computed as
follows: \[ERS = K - 1 + k.\]

The transformations outlined above are suitable only for ordered
probability vectors, that is, for informed hypotheses in binomial and
multinomial models that only feature inequality constraints. However,
when informed hypotheses also feature equality constrained parameters,
as well as parameters that are free to vary we need to modify the
formula. Specifically, to determine the lower bounds for each parameter,
we need to take into account for each element \(\theta_k\) the number of
equality constrained parameters that are collapsed within this element
(denoted as \(e_k\)):

\begin{align}
  l_k = \left.
  \begin{cases}
      0 & \text{if } k = 1 \\
      \frac{\theta_{k - 1}}{e_{k-1}} \times e_k & \text{if } 1 < k < K.
  \end{cases}
    \right.
\end{align} The upper bound for parameters in the binomial models still
remains \(1\). To determine the upper bound for multinomial models we
must additionally for each element \(\theta_k\) take into account the
number of free parameters that share common upper and lower bounds
(denoted with \(f_k\)). The upper bound is then defined as:

\begin{align}
  u_k = \left.
  \begin{cases}
      \cfrac{1 - (f_k \times l_k)}{K} & \text{if } k = 1 \\
     \left( \cfrac{1 - \sum_{i < k} \theta_i - (f_k \times l_k)}{ERS} \right) \times e_k & \text{if } 1 < k < K \text{ and } u_k \geq \text{max}(\theta_{i < k}), \\
    \left( 2 \times \left( \cfrac{1 - \sum_{i < k} \theta_i - (f_k \times l_k)}{ERS} \right) - \text{max}(\theta_{i < k}) \right)  \times e_k & \text{if } 1 < k < K \text{ and } u_k < \text{max}(\theta_{i < k}).
  \end{cases}
    \right.
\end{align} The elements in the remaining stick are then computed as
follows \[ERS = e_k + \sum_{j > k} e_j \times f_j.\] The rationale
behind these modifications will be described in more detail in the
following sections. In \textbf{multibridge}, information that is
relevant for the transformation of the parameter vectors is stored in
the generated \texttt{restriction\_list} which is returned by the main
functions \texttt{binom\_bf\_informed} and \texttt{mult\_bf\_informed}
but can also be generated separately with the function
\texttt{generate\_restriction\_list}. This restriction list features the
sublist \texttt{inequality\_constraints} which encodes the number of
equality constraints collapsed in each parameter in
\texttt{nr\_mult\_equal}. Similarly the number of free parameters that
share a common bounds are encoded under \texttt{nr\_mult\_free}.

\hypertarget{equality-constrained-parameters}{%
\paragraph{Equality Constrained
Parameters}\label{equality-constrained-parameters}}

In cases where informed hypotheses feature a mix of equality and
inequality constrained parameters, we compute the corresponding Bayes
factor \(\text{BF}_{re}\), by multiplying the individual Bayes factors
for both constrait types with each other:

\[
\text{BF}_{re}
= \text{BF}_{1e} \times \text{BF}_{2e} \mid \text{BF}_{1e},
\] where the subscript \(1\) denotes the hypothesis that only features
equality constraints and the subscript \(2\) denotes the hypothesis that
only features inequality constraints. To receive
\(\text{BF}_{2e} \mid \text{BF}_{1e}\), we collapse in the constrained
prior and posterior distributions all equality constrained parameters
into one category which has implications on the performed
transformations.

When transforming the samples from these distributions, we need to
account for the fact that the inequality constraints imposed under the
original parameter values might not hold for the collapsed parameters.
Consider, for instance, a multinomial model in which we specify the
following informed hypothesis
\[\mathcal{H}_r: \theta_1 \leq \theta_2 = \theta_3 = \theta_4 \leq \theta_5 \leq \theta_6,\]
where samples from the encompassing distribution take the values
\((0.05, 0.15, 0.15, 0.15, 0.23, 0.27)\). For these parameter values the
inequality constraints hold since \(0.05\) is smaller than \(0.15\),
\(0.23\) and \(0.27\). However, the same constraint does not hold when
we collapse the categories \(\theta_2\), \(\theta_3\), and \(\theta_4\)
into \(\theta_*\). That is, the collapsed parameter
\(\theta_* = 0.15 + 0.15 + 0.15 = 0.45\) is now larger than \(0.23\) and
\(0.27\). In general, to determine the lower bound for a given parameter
\(\theta_k\) we thus need to take into account both the number of
collapsed categories in the preceding parameter \(e_{k-1}\) as well as
the number of collapsed categories in the current parameter \(e_{k}\).
In the example above, this means that to determine the lower bound for
\(\theta_*\) we multiply the preceding value \(\theta_1\) by three, such
that the lower bound is \(0.05 \times 3 = 0.15\). In addition, to
determine the lower bound of \(\theta_5\) we divide the preceding value
\(\theta_*\) by three, that is, \(0.6/3 = 0.2\). In general, lower
bounds for the parameters need to be adjusted as follows: \begin{align}
  l_k = \left.
  \begin{cases}
      0 & \text{if } k = 1 \\
      \frac{\theta_{k - 1}}{e_{k-1}} \times e_k & \text{if } 1 < k < K,
  \end{cases}
    \right.
\end{align} where \(e_{k-1}\) and \(e_k\) refer to the number of
equality constrained parameters that are collapsed in \(\theta_{k - 1}\)
and \(\theta_{k}\), respectively. Similarly, to determine the upper
bound for a given parameter value, we need to multiple the upper bound
the number of equality constrained parameters within the current
constraint:

\begin{align}
  u_k = \left.
  \begin{cases}
      \cfrac{1}{ERS} \times e_k & \text{if } k = 1 \\
      \cfrac{1 - \sum_{i < k} \theta_i}{ERS} \times e_k & \text{if } 1 < k < K,
  \end{cases}
    \right.
\end{align} where \(1 - \sum_{i < k} \theta_i\) represents the length of
the remaining stick and the number of elements in the remaining stick
are computed as follows: \(ERS = \sum_k^{K} e_k\). For the example
above, the upper bound for \(\theta_*\) is
\(\cfrac{1 - 0.05}{5} \times 3 = 0.57\). The upper bound for
\(\theta_5\) is then \(\cfrac{(1 - 0.05 - 0.45)}{2} \times 1 = 0.25\).

\hypertarget{corrections-for-free-parameters}{%
\paragraph{Corrections for Free
Parameters}\label{corrections-for-free-parameters}}

Different adjustments are required for a sequence of inequality
constrained parameters that share upper and lower bounds. Consider, for
instance, a multinomial model in which we specify the informed
hypothesis
\[\mathcal{H}_r: \theta_1 \leq \theta_2, \theta_3 \leq \theta_4.\] This
hypothesis specifies that \(\theta_2\) and \(\theta_3\) have the shared
lower bound \(\theta_1\) and the shared upper bound \(1\), however,
\(\theta_2\) can be larger than \(\theta_3\) or vice versa. To integrate
these cases within the stick-breaking approach one must account for
these potential changes of order. For these cases, the lower bounds for
the parameters remain unchanged. To determine the upper bounds, we need
to subtract for each \(\theta_k\) from the length of the remaining stick
the lower bounds of all parameters that share common bounds with
\(\theta_k\) and that have not yet been accounted for in the
transformation: \begin{align}
  u_k = \left.
  \begin{cases}
      \cfrac{1 - (f_k \times l_k)}{K} & \text{if } k = 1 \\
      \cfrac{1 - \sum_{i < k} \theta_i - (f_k \times l_k)}{ERS} & \text{if } 1 < k < K,
  \end{cases}
    \right.
\end{align}

where \(f_k\) represents the number of free parameters that share common
upper and lower bounds with \(\theta_k\) and that have been not yet been
accounted for. Here, the number of elements in the remaining stick is
defined as the number of all parameters that are larger than
\(\theta_k\): \(ERS = 1 + \sum_{j > k} f_j\). To illustrate this
correction, assume that samples from the encompassing distribution take
the values \((0.15, 0.3, 0.2, 0.35)\). The upper bound for \(\theta_1\)
is simply \(\nicefrac{1}{4}\). For \(\theta_2\), we need to take into
account that \(\theta_2\) and \(\theta_3\) share upper and lower bounds.
Thus, to compute the upper bound for \(\theta_2\), we subtract from the
length of the remaining stick the lower bound of \(\theta_3\):
\(\cfrac{1 - 0.15 - (0.15 \times 1)}{2} = 0.35\).

A further correction is required, if a preceding free parameter (i.e., a
free parameter that was already accounted for in the stick) is larger
than the upper bound of the current parameter. For instance, in our
example the upper bound for \(\theta_3\) would be
\(\cfrac{1 - 0.15 - 0.3}{2} = 0.275\), but the preceding free parameter
is \(0.3\). However, if \(\theta_3\) would actually take on the value
\(0.275\), then \(\theta_4\) would have to be \(0.275\) as well, which
would violate the constraint (i.e.,
\(0.15 \leq 0.3, 0.275 \nleq 0.275\)). In these cases, the upper bound
needs to be corrected downwards. To do this, we subtract the difference
between the largest preceding free parameter in the sequence with the
current upper bound. Thus, if \(u_k < \text{max}(\theta_{i < k})\), the
upper bound becomes: \begin{align}
u_k &= u_k - (\text{max}(\theta_{i < k}) - u_k) \\
    &= 2 \times u_k - \text{max}(\theta_{i < k}).
\end{align} For our example the corrected upper bound for \(\theta_3\)
would become \(2*0.275 - 0.3 = 0.25\) which secures the proper ordering
for the remainder of the parameters: if \(\theta_3\) would take on the
value \(0.25\), \(\theta_4\) would be \(0.3\) which would be in
accordance with the constraint, that is,
\(0.15 \leq 0.3, 0.25 \leq 0.3\).

\hypertarget{references}{%
\subsection{References}\label{references}}

\begingroup
\setlength{\parindent}{-0.5in}
\setlength{\leftskip}{0.5in}

\hypertarget{refs}{}
\begin{cslreferences}
\leavevmode\hypertarget{ref-frigyik2010introduction}{}%
Frigyik, B. A., Kapila, A., \& Gupta, M. R. (2010). \emph{Introduction
to the Dirichlet distribution and related processes}. Department of
Electrical Engineering, University of Washington.

\leavevmode\hypertarget{ref-sarafoglou2020evaluatingPreprint}{}%
Sarafoglou, A., Haaf, J. M., Ly, A., Gronau, Q. F., Wagenmakers, E., \&
Marsman, M. (2020). Evaluating multinomial order restrictions with
bridge sampling. \emph{PsyArXiv}. Retrieved from
\url{https://psyarxiv.com/bux7p/}

\leavevmode\hypertarget{ref-stan2020}{}%
Stan Development Team. (2020). \emph{Stan modeling language user's guide
and reference manual, version 2.23.0}. R Foundation for Statistical
Computing. Retrieved from \url{http://mc-stan.org/}
\end{cslreferences}

\endgroup
\end{appendix}
